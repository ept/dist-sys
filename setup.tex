\PassOptionsToPackage{hyphens}{url}
\PassOptionsToPackage{colorlinks,linkcolor=darkblue,citecolor=darkblue,urlcolor=darkblue}{hyperref}
\usepackage[utf8]{inputenc}
\usepackage[english]{babel}
\usepackage[hyphens]{url}
\usepackage{doi}
\usepackage{hyperref}
\usepackage{amsmath}
\usepackage{amsthm}
\usepackage{tikz}
\usepackage{natbib}

% The minted package does source code syntax highlighting using Pygments: https://pygments.org/
% Pygments must be installed on your system, e.g. using `pip3.8 install Pygments`.
% pdflatex must be run with option -shell-escape so that it can run pygmentize.
% To allow the document to be built on systems without Pygments, commit the files in the
% _minted-*/ directories to git. Use option finalizecache=true to update the cached
% syntax-highlighted listings, and use option frozencache=true to use that cache.
% With frozencache=true, the -shell-escape option is no longer needed.
\usepackage{minted}

\usetikzlibrary{arrows.meta}
\tikzstyle{bigarrow}=[very thick,-{Classical TikZ Rightarrow[length=2mm]}]
\tikzstyle{doublebigarrow}=[very thick,{Classical TikZ Rightarrow[length=2mm]}-{Classical TikZ Rightarrow[length=2mm]}]
\tikzstyle{messageloss}=[very thick,-{Rays[length=8mm,red,line width=4pt]}]

\urlstyle{sf}
\newcounter{inlineslides}

\newcommand{\inlineslide}[1]{
    \refstepcounter{inlineslides}
    \vspace{1em}\pagebreak[2]\noindent
    \fbox{
      \begin{minipage}[t]{9cm}
        \begin{center}
          \includeslide[width=9cm,page=0]{#1} \\
        \end{center}
      \end{minipage}
    }
    \hspace{1em}\textsf{Slide \theinlineslides} \par\vspace{1em}\pagebreak[2]
}
\def\inlineslidesautorefname{Slide}%

\definecolor{darkblue}{rgb}{0,0,0.7}
\definecolor{lightgrey}{rgb}{0.95,0.95,0.95}

% Exercises and solution notes. See also dist-sys-notes.tex
\newtheorem{exercise}{Exercise}
\newcommand\supervision[2]{\begin{exercise}#1\end{exercise}}
